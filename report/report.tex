 \documentclass{article}
\begin{document}

\title{Ontleden en genereren van taal met een context-free grammar}
\author{Janne Spijkervet}
\maketitle

\section{Introductie}
Voor het ontleden en generen van taal wordt in de computerwetenschappen gebruik gemaakt van \emph{natural language processing}.

In dit onderzoek wordt onderzocht of met een Context-Free Grammar de zinnen van het corpus kunnen worden beschreven en gegenereerd. Daarna wordt het resultaat van deze grammar vergeleken met de resultaten van een nieuwe grammar die wordt gemaakt met een \emph{feature grammar} en een \emph{probabilistic context-free grammar}.

Als corpus is het eerste hoofdstuk van het zevende boek in de Harry Potter reeks van J.K. Rowling gekozen, "Harry Potter and the Deathly Hallows". De taal van het corpus is Engels.




\section{Beschrijving}

Je beschrijft de context van het onderzoek. Wat is er onderzocht en waarom? Meestal kun je het opdelen in twee onderdelen: onderwerp en onderzoeksvraag.

Bij de beschrijving van het onderwerp geef je aan wat de bredere context is, de relevantie, en eerdere bevindingen. Geef in enkele zinnen aan wat de algemene achtergrond van het rapport is. Als het een opdracht is bij een practicum, beschrijf je hier in je eigen woorden de instructies. Hierbij staat vaak een groter probleem centraal; aan de oplossing lever je met je onderzoek een kleine bijdrage. Informatie die nodig is om de rest van het rapport te kunnen begrijpen, zoals definities van belangrijke concepten, geef je ook hier. Daarbij geef je aan waarom dit onderwerp belangrijk is, dan wel op wetenschappelijk of maatschappelijk niveau. In dit geheel kun je bevindingen uit eerdere onderzoek beschrijven ter onderbouwing.

Na de omschrijving van het onderwerp geef je expliciet aan welke vraag je in dit onderzoek gaat beantwoorden. Deze onderzoeksvraag is het belangrijkste onderdeel van het rapport en vormt de rode draad voor de andere onderdelen. Een heldere onderzoeksvraag biedt houvast bij het schrijven. Alle gegevens die je in het rapport vermeldt, dienen tot het geven van een antwoord op de onderzoeksvraag. Zonder een duidelijke vraag zal het moeilijk zijn om tot een conclusie te komen.

Indien je specifieke verwachtingen hebt over het antwoord op je onderzoeksvraag, dan dien je deze ook te vermelden. Daarbij horen de verwachtingen onderbouwd te zijn: als het goed is, zouden deze logisch moeten volgen uit de eerdere bevindingen uit de literatuur die je hiervoor hebt beschreven.

Hieronder volgen fragmenten van een technisch rapport, gebasseerd op rapporten die eerstejaars studenten afgelopen jaren geschreven hebben. Dit voorbeeld illustreert hoe de verschillende onderdelen in één sectie terugkomen. Het is op andere aspecten geen perfect voorbeeld dat letterlijk overgenomen dient te worden.



\end{document}
